\chapter{Debug}
\label{chap:debug}

\rvcore has full support for software breakpoints (\instr{ebreak}), access to
general-purpose and control and status registers via a debug port. It is also
possible to halt the core from the debug port and put it into single-stepping
mode. Similarly when an interrupt occurs, instead of jumping to the interrupt
handler the core can trap to an attached debugger.

The debug port uses the following interface:

\begin{table}[H]
 \caption{Debug Signals}
 \label{tab:debug_signals}
  \begin{tabularx}{\textwidth}{@{}llX@{}} \toprule
    \textbf{Signal}                 & \textbf{Direction} & \textbf{Description} \\ \toprule
    \signal{dbginf\_strobe\_i}      & \textbf{input}     & Command request \\ \hline
    \signal{dbginf\_we\_i}          & \textbf{input}     & Write Enable \\ \hline
    \signal{dbginf\_addr\_i[15:0]}  & \textbf{input}     & Address \\ \hline
    \signal{dbginf\_data\_i[31:0]}  & \textbf{input}     & Input data \\ \hline
    \signal{dbginf\_data\_o}        & \textbf{output}    & Output data \\ \hline
    \signal{dbginf\_ack\_o}         & \textbf{output}    & Command was executed \\ \hline
    \signal{dbginf\_stall\_i}       & \textbf{input}     & Stall the core \\ \hline
    \signal{dbginf\_bp\_o}          & \textbf{output}    & Breakpoint hit \\ \bottomrule
  \end{tabularx}
\end{table}

This interface is natively supported by the advanced debug bridge that is used
by \pulp and \pulpino, see also the documentation of this bridge.


\section{Debug Address Map}

This debug address map is not optimal and should be changed!
See the OpenSoC debug project for a proposal for a better address map.

\begin{table}[H]
 \caption{Control and Status Register Map}
 \label{tab:debug_map}
 \centering\begin{tabularx}{\linewidth}{@{}|cc|c|l|X|@{}} \toprule
   \multicolumn{2}{|c|}{\textbf{Dbginf Addr [15:0]}} & \textbf{Hex} & \textbf{Name} & \textbf{Description} \\ \hline
   Grp [15:11] & Addr [10:0] & & & \\ \toprule
   \texttt{0\_0001} & \texttt{000\_000X\_XXXX} & \texttt{0x0800 - 0x081F} & GPR   & General-Purpose Registers \\ \hline
   \texttt{0\_0110} & \texttt{000\_0000\_0000} & \texttt{0x3000 - 0x3014} & Debug & Debug Registers \\ \hline
   \texttt{?\_????} & \texttt{XXX\_XXXX\_XXXX} & \texttt{               } & CSR   & Everything else is mapped to CSR \\ \bottomrule
  \end{tabularx}
\end{table}


\subsection{Debug Register: DMR1}

\textbf{CSR Address:} \texttt{0x3010} \\
\textbf{Reset Value:} \texttt{0x0000\_0000} \\
\begin{figure}[H]
  \centering
  \begin{bytefield}[endianness=big,bitheight=60pt]{32}
    \bitheader{31,23,22,0} \\
    \bitbox{9}{ Unused }
    \bitbox{1}{\rotatebox{90}{\tiny Single-Stepping    }}
    \bitbox{22}{ Unused }
  \end{bytefield}
  \caption{DMR1}
\end{figure}

Single-stepping activates single-stepping mode, meaning the core traps to the
debugger after one instruction has been executed.

\subsection{Debug Register: DSR}

\textbf{CSR Address:} \texttt{0x3014} \\
\textbf{Reset Value:} \texttt{0x0000\_0000} \\
\begin{figure}[H]
  \centering
  \begin{bytefield}[endianness=big,bitheight=60pt]{32}
    \bitheader{31,8,7,6,0} \\
    \bitbox{24}{ Unused }
    \bitbox{1}{\rotatebox{90}{\tiny INTE   }}
    \bitbox{1}{\rotatebox{90}{\tiny IIE    }}
    \bitbox{6}{ Unused }
  \end{bytefield}
  \caption{DMR1}
\end{figure}

\signal{IIE} stands for illegal instruction exception enabled. A value of
\signal{1} means trap to the debugger when an illegal instruction is
encountered.

\signal{INTE} stands for interrupt enabled. A value of \signal{1} means trap to
the debugger when an interrupt is encountered.
