\chapter{Exceptions and Interrupts}
\label{chap:exceptions}

\rvcore supports vectorized interrupts, exceptions on illegal instructions and
exceptions on load and store instructions to invalid addresses.


\begin{table}[H]
 \caption{Interrupt/exception offset vector table}
 \label{tab:exc_table}
 \centering\begin{tabular}{@{}ll@{}} \toprule
    \textbf{Address} & \textbf{Description} \\ \toprule
    \signal{0x00} - \signal{0x0000\_007C} & Interrupts 0 - 31 \\ \hline
    \signal{0x80} & Reset \\ \hline
    \signal{0x84} & Illegal Instruction \\ \hline
    \signal{0x88} & \instr{ECALL} instruction executed \\ \hline
    \signal{0x8C} & LSU error (invalid memory access) \\ \bottomrule
  \end{tabular}
\end{table}

The instruction addresses in Table~\ref{tab:exc_table} are considered as an
offset to the boot address given to the core. Specifically the core jumps to
address \signal{$\{$boot\_addr[31:8], offset[7:0]$\}$} when encountering an
exception/interrupt.


\section{Interrupts}
\rvcore uses vectorized interrupts, specifically it provides 32 separate
interrupt lines. Interrupts can only be enabled/disabled on a global basis and
not individually. It is assumed that there is an event/interrupt controller
outside of the core that performs masking and buffering of the interrupt lines.
The global interrupt enable is done via the CSR register \signal{MSTATUS}.

\section{Exceptions}

The illegal instruction exception, the load and store invalid memory access
exceptions and ecall instruction exceptions can not be disabled and are always
active.

The illegal instruction exception and the load and store invalid memory access
exceptions are precise exceptions, i.e. the value of \signal{MEPC} will be the
instruction address that caused it.

\section{Handling}

\rvcore does support nested interrupt/exception handling. Exceptions inside
interrupt/exception handlers cause another exception, thus exceptions during the
critical part of your exception handlers, i.e. before having saved the
\signal{MEPC} and \signal{MESTATUS} registers, will cause those register to be
overwritten.
Interrupts during interrupt/exception handlers are disabled by default, but can
be explicitely enabled if desired.

Upon executing an \instr{eret} instruction, the core jumps to the program
counter saved in the CSR register \signal{MEPC} and restores the value of
register \signal{MESTATUS} to \signal{MSTATUS}. When entering an
interrupt/exception handler, the core sets \signal{MEPC} to the current program
counter and saves the current value of \signal{MSTATUS} in \signal{MESTATUS}.
