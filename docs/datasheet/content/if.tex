\chapter{Instruction Fetch}
\label{chap:if}

The instruction fetcher of the core is able to supply one instruction to the ID
stage per cycle if the instruction cache or the instruction memory is able to
serve one instruction after one cycle.
The instruction address must be half-word-aligned. It is not possible to jump to
instruction addresses that have the LSB bit set.

For optimal performance and timing closure reasons, a prefetcher is used to
fetch instructions. There are two prefetch flavors available:
\begin{itemize}
  \item 32-Bit Word prefetcher. It stores the fetched words in a FIFO with three
    entries.
  \item 128-Bit Cache line prefetcher. It stores one 128-bit wide cache line
    plus 32-bit to allow for cross-cache line misaligned instructions.
\end{itemize}

Table~\ref{tab:instr_signals} describes the signals that are used to fetch
instructions. This interface is a simplified version that is used by the
LSU that is described in Chapter~\ref{chap:lsu}. The difference is that no
writes are possible and thus it needs less signals.

\begin{table}[H]
 \caption{Instruction Fetch Signals}
 \label{tab:instr_signals}
  \begin{tabularx}{\textwidth}{@{}llX@{}} \toprule
    \textbf{Signal}                & \textbf{Direction} & \textbf{Description} \\ \toprule
    \signal{instr\_req\_o}         & \textbf{output}    & Request ready, must stay high until \signal{instr\_gnt\_i} is high for one cycle \\ \hline
    \signal{instr\_addr\_o[31:0]}  & \textbf{output}    & Address \\ \hline
    \signal{instr\_rdata\_i[31:0]} & \textbf{input}     & Data read from memory \\ \hline
    \signal{instr\_rvalid\_i}      & \textbf{input}     & \signal{instr\_rdata\_i} is valid now for this cycle. When \signal{instr\_rvalid\_i} is high, another request can be sent. \\ \hline
    \signal{instr\_gnt\_i}         & \textbf{input}     & The instruction cache accepted the request. The \signal{instr\_addr\_o} may be change in the next cylce. \\ \bottomrule
  \end{tabularx}
\end{table}


\section{Protocol}
The protocol used to communicate with the instruction cache or the instruction
memory is the same as the protocl used by the LSU. See the description of the
LSU in Chapter~\ref{sec:lsu_protocol} for details about the protocol.
