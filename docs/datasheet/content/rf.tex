\chapter{Register File}
\label{chap:rf}

\rvcore has 31 $\times$ 32-bit wide registers which form registers \signal{x1} to
\signal{x31}. Register \signal{x0} is statically bound \signal{0} and can only be
read and not written, it does not contain any sequential logic.

There are two flavors of register file available:

\begin{enumerate}
  \item Latch-based
  \item Flip-flop based
\end{enumerate}

While the latch-based register file is recommended for ASICs, the flip-flop
based register file is recommended for FPGA synthesis, although both are
compatible with either synthesis target.
Note the flip-flop based register file is significantly larger than the
latch-based register-file for an ASIC implementation.

\section{Latch-based Register File}
The latch based register file contains manually instantiated clock gating cells
to keep the clock inactive when the latches are not written.

It is assumed that there is a clock gating cell for the target technology that
is wrapped in a module called \signal{cluster\_clock\_gating} and has the following
ports:
\begin{itemize}
  \item \signal{clk\_i}: Clock Input
  \item \signal{en\_i}: Clock Enable Input
  \item \signal{test\_en\_i}: Test Enable Input (activates the clock even though
  \signal{en\_i} is not set)
  \item \signal{clk\_o}: Gated Clock Output
\end{itemize}
