\chapter{Overview}

\rvcore is a 4-stage in-order \riscv CPU. The ISA of \rvcore was extended to
also support multiple additional instructions including hardware loops,
post-increment load and store instructions and additional ALUinstructions that
are not part of the standard \riscv ISA.

Figure~\ref{fig:ri5cy_overview} shows a block diagram of the core.

\begin{figure}[H]
  \centering
  \includegraphics[width=0.9\textwidth]{./figures/ri5cy_overview}
  \caption{\rvcore Overview.}
  \label{fig:ri5cy_overview}
\end{figure}


\section{Supported Instruction Set}

\rvcore supports the following instructions:

\begin{itemize}
  \item Full support for RV32I Base Integer Instruction Set
  \item Full support for RV32C Standard Extension for Compressed Instructions
  \item Partial support for RV32M Standard Extension for Integer Multiplication
    and Division \\
        Only the \instr{mul} instruction is supported.
  \item PULP specific extensions \\
        \begin{itemize}
          \item Hardware Loops, see Chapter~\ref{chap:hwloop}
          \item ALU extensions, see Chapter~\ref{chap:aluext}
          \item Multiply-Accumulate extensions, see Chapter~\ref{chap:mac}
          \item Post-Incrementing load and stores, see Chapter~\ref{chap:lsu}
        \end{itemize}
\end{itemize}

\section{ASIC Synthesis}
ASIC synthesis is supported for \rvcore. The whole design is completely
synchronous and uses positive-edge triggered flip-flops, except for the register
file, where there is an option to use latches instead of flip-flops. See
Chapter~\ref{chap:rf} for more details about the register file. The core
occupies an area of about 35~kGE when the latch based register file is used.

\section{FPGA Synthesis}
FPGA synthesis is supported for \rvcore when the flip-flop based register file
is used. Since latches are not well supported on FPGAs, it is crucial to select
the flip-flop based register file.
