\chapter{PULP Hardware Loop Extensions}
\label{chap:hwloop}

For increased efficiency of small loops, RI5CY supports hardware loops.
Hardware loops make it possible to execute a piece of code multiple times,
without the overhead of branches or updating a counter. Hardware loops involve
zero stall cycles for jumping to the first instruction of a loop.

A hardware loop is defined by its start address (pointing to the first
instruction in the loop), its end address (pointing to the instruction that
will be executed last in the loop) and a counter that is decremented every time
the loop body is executed. RI5CY contains two hardware loop register sets, each
of them can store these three values.
If the end address of the two hardware loops is identical, loop 0 has higher
priority and only the loop counter for hardware loop 0 is decremented.
As soon as the counter of loop 0 reaches 1 at an end address, meaning it is
decremented to 0 now, loop 1 gets active too. In this case, both counters will
be decremented and the core jumps to the start of loop 1.

The instructions described below are used to setup the hardware loop registers.
Note that the minimum loop size is two instructions.

For debugging and context switches the hardware loop registers are mapped into
the CSR address space and thus it is possible to read and write them via
\instr{csrr} and \instr{csrw} instructions.


\section{Instructions}

\subsection{lp.starti L, uimmL}
Sets the start address of a hardware loop.

\begin{center}
  \begin{bytefield}[endianness=big,bitwidth=1.3em]{32}
    \bitheader{31,20,19,15,14,12,11,7,6,0} \\
    \bitbox{12}{ uimmL[11:0]                      }
    \bitbox{5}{ rs1                               }
    \bitbox{3}{ funct3                            }
    \bitbox{4}{ 0000                              }
    \bitbox{1}{ L                                 }
    \bitbox{7}{ opcode                            } \\

    \bitbox[]{12}{ uimmL[11:0] }
    \bitbox[]{5}{ 00000 }
    \bitbox[]{3}{ 000 }
    \bitbox[]{4}{ 0000 }
    \bitbox[]{1}{ L }
    \bitbox[]{7}{ 111 1011 }
  \end{bytefield}
\end{center}
\textbf{Operation:} \texttt{lpstart[L] = pc + (Zext(uimmL[11:0]) << 1)}


\subsection{lp.endi L, uimmL}
Sets the end address of a hardware loop.

\begin{center}
  \begin{bytefield}[endianness=big,bitwidth=1.3em]{32}
    \bitheader{31,20,19,15,14,12,11,7,6,0} \\
    \bitbox{12}{ uimmL[11:0]                      }
    \bitbox{5}{ rs1                               }
    \bitbox{3}{ funct3                            }
    \bitbox{4}{ 0000                              }
    \bitbox{1}{ L                                 }
    \bitbox{7}{ opcode                            } \\

    \bitbox[]{12}{ uimmL[11:0] }
    \bitbox[]{5}{ 00000 }
    \bitbox[]{3}{ 001 }
    \bitbox[]{4}{ 0000 }
    \bitbox[]{1}{ L }
    \bitbox[]{7}{ 111 1011 }
  \end{bytefield}
\end{center}
\textbf{Operation:} \texttt{lpend[L] = pc + (Zext(uimmL[11:0]) << 1)}


\subsection{lp.count L, rs1}
Sets the number of iterations of a hardware loop.

\begin{center}
  \begin{bytefield}[endianness=big,bitwidth=1.3em]{32}
    \bitheader{31,20,19,15,14,12,11,7,6,0} \\
    \bitbox{12}{ uimmL[11:0]                      }
    \bitbox{5}{ rs1                               }
    \bitbox{3}{ funct3                            }
    \bitbox{4}{ 0000                              }
    \bitbox{1}{ L                                 }
    \bitbox{7}{ opcode                            } \\

    \bitbox[]{12}{ 0000 0000 0000}
    \bitbox[]{5}{ src1 }
    \bitbox[]{3}{ 010 }
    \bitbox[]{4}{ 0000 }
    \bitbox[]{1}{ L }
    \bitbox[]{7}{ 111 1011 }
  \end{bytefield}
\end{center}
\textbf{Operation:} \texttt{lpcount[L] = rs1}


\subsection{lp.counti L, uimmL}
Sets the number of iterations of a hardware loop.

\begin{center}
  \begin{bytefield}[endianness=big,bitwidth=1.3em]{32}
    \bitheader{31,20,19,15,14,12,11,7,6,0} \\
    \bitbox{12}{ uimmL[11:0]                      }
    \bitbox{5}{ rs1                               }
    \bitbox{3}{ funct3                            }
    \bitbox{4}{ 0000                              }
    \bitbox{1}{ L                                 }
    \bitbox{7}{ opcode                            } \\

    \bitbox[]{12}{ uimmL[11:0] }
    \bitbox[]{5}{ 00000 }
    \bitbox[]{3}{ 011 }
    \bitbox[]{4}{ 0000 }
    \bitbox[]{1}{ L }
    \bitbox[]{7}{ 111 1011 }
  \end{bytefield}
\end{center}
\textbf{Operation:} \texttt{lpcount[L] = Zext(uimmL[11:0])}


\subsection{lp.setup L, rs1, uimmL}
Sets up a hardware loop in one instruction. This instruction assumes that the
next instruction is the start address of the loop.

\begin{center}
  \begin{bytefield}[endianness=big,bitwidth=1.3em]{32}
    \bitheader{31,20,19,15,14,12,11,7,6,0} \\
    \bitbox{12}{ uimmL[11:0]                      }
    \bitbox{5}{ rs1                               }
    \bitbox{3}{ funct3                            }
    \bitbox{4}{ 0000                              }
    \bitbox{1}{ L                                 }
    \bitbox{7}{ opcode                            } \\

    \bitbox[]{12}{ uimmL[11:0] }
    \bitbox[]{5}{ src1 }
    \bitbox[]{3}{ 100 }
    \bitbox[]{4}{ 0000 }
    \bitbox[]{1}{ L }
    \bitbox[]{7}{ 111 1011 }
  \end{bytefield}
\end{center}
\textbf{Operation:} \texttt{lpstart[L] = pc + 4; lpend[L] = pc + Zext(uimmL[11:0]); lpcount[L] = rs1}


\subsection{lp.setupi L, uimmS, uimmL}
Sets up a hardware loop in one instruction. This instruction assumes that the
next instruction is the start address of the loop. The number of iterations is
given as an immediate.

\begin{center}
  \begin{bytefield}[endianness=big,bitwidth=1.3em]{32}
    \bitheader{31,20,19,15,14,12,11,7,6,0} \\
    \bitbox{12}{ uimmL[11:0]                      }
    \bitbox{5}{ uimmS                             }
    \bitbox{3}{ funct3                            }
    \bitbox{4}{ 0000                              }
    \bitbox{1}{ L                                 }
    \bitbox{7}{ opcode                            } \\

    \bitbox[]{12}{ uimmL[11:0] }
    \bitbox[]{5}{ uimmS }
    \bitbox[]{3}{ 101 }
    \bitbox[]{4}{ 0000 }
    \bitbox[]{1}{ L }
    \bitbox[]{7}{ 111 1011 }
  \end{bytefield}
\end{center}
\textbf{Operation:} \texttt{lpstart[L] = pc + 4; lpend[L] = pc + Zext(uimmL[11:0]); lpcount[L] = Zext(uimmS)}


\section{CSR Mapping}

\begin{table}[H]
 \caption{Control and Status Register Map}
 \centering\begin{tabularx}{\linewidth}{@{}|cc|c|l|l|X|@{}} \toprule
   \multicolumn{2}{|c|}{\textbf{CSR Address}} & \textbf{Hex} & \textbf{Name} & \textbf{Access}  & \textbf{Description} \\ \hline
   \textbf{[11:6]} & \textbf{[5:0]} & & & & \\ \toprule
   011110 & 110000  & 0x7B0 & lpstart[0] & R/W & Hardware Loop 0 Start \\ \hline
   011110 & 110001  & 0x7B1 & lpend[0]   & R/W & Hardware Loop 0 End \\ \hline
   011110 & 110010  & 0x7B2 & lpcount[0] & R/W & Hardware Loop 0 Counter \\ \hline
   011110 & 110100  & 0x7B4 & lpstart[1] & R/W & Hardware Loop 1 Start \\ \hline
   011110 & 110101  & 0x7B5 & lpend[1]   & R/W & Hardware Loop 1 End \\ \hline
   011110 & 110110  & 0x7B6 & lpcount[1] & R/W & Hardware Loop 1 Counter \\ \bottomrule
  \end{tabularx}
\end{table}
