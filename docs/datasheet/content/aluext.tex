\chapter{PULP ALU Extensions}
\label{chap:aluext}

\rvcore supports advanced ALU operations that allow to perform multiple
instructions that are specified in the base instruction set in one single
instruction and thus increases efficiency of the core.
For example those instructions include zero-/sign-extension instructions for
8-bit and 16-bit operands, simple bit manipulation/counting instructions and
min/max/avg instructions.


\section{Instructions}

\subsection{p.avg rD, rs1, rs2}
Performs an arithmetic right shift after the addition of rs1 and rs2.

\begin{center}
  \begin{bytefield}[endianness=big,bitwidth=1.3em]{32}
    \bitheader{31,25,24,20,19,15,14,12,11,7,6,0} \\
    \bitbox{7}{ funct7                            }
    \bitbox{5}{ rs2                               }
    \bitbox{5}{ rs1                               }
    \bitbox{3}{ funct3                            }
    \bitbox{5}{ rd                                }
    \bitbox{7}{ opcode                            } \\

    \bitbox[]{7}{ 000 0010}
    \bitbox[]{5}{ src2 }
    \bitbox[]{5}{ src1 }
    \bitbox[]{3}{ 000 }
    \bitbox[]{5}{ dest }
    \bitbox[]{7}{ 011 0011 }
  \end{bytefield}
\end{center}
\textbf{Operation:} \texttt{rD = (rs1 + rs2) >> 1}


\subsection{p.avgu rD, rs1, rs2}
Performs a logical right shift after the addition of rs1 and rs2.

\begin{center}
  \begin{bytefield}[endianness=big,bitwidth=1.3em]{32}
    \bitheader{31,25,24,20,19,15,14,12,11,7,6,0} \\
    \bitbox{7}{ funct7                            }
    \bitbox{5}{ rs2                               }
    \bitbox{5}{ rs1                               }
    \bitbox{3}{ funct3                            }
    \bitbox{5}{ rd                                }
    \bitbox{7}{ opcode                            } \\

    \bitbox[]{7}{ 000 0010}
    \bitbox[]{5}{ src2 }
    \bitbox[]{5}{ src1 }
    \bitbox[]{3}{ 001 }
    \bitbox[]{5}{ dest }
    \bitbox[]{7}{ 011 0011 }
  \end{bytefield}
\end{center}
\textbf{Operation:} \texttt{rD = (rs1 + rs2) >> 1}


\subsection{p.slet rD, rs1, rs2}

Performs an signed smaller than or equal comparison between rs1 and rs2.

\begin{center}
  \begin{bytefield}[endianness=big,bitwidth=1.3em]{32}
    \bitheader{31,25,24,20,19,15,14,12,11,7,6,0} \\
    \bitbox{7}{ funct7                            }
    \bitbox{5}{ rs2                               }
    \bitbox{5}{ rs1                               }
    \bitbox{3}{ funct3                            }
    \bitbox{5}{ rd                                }
    \bitbox{7}{ opcode                            } \\

    \bitbox[]{7}{ 000 0010}
    \bitbox[]{5}{ src2 }
    \bitbox[]{5}{ src1 }
    \bitbox[]{3}{ 010 }
    \bitbox[]{5}{ dest }
    \bitbox[]{7}{ 011 0011 }
  \end{bytefield}
\end{center}
\textbf{Operation:} \texttt{rD = (rs1 <= rs2) ? -1 : 0}


\subsection{p.sletu rD, rs1, rs2}

Performs an unsigned smaller than or equal comparison between rs1 and rs2.

\begin{center}
  \begin{bytefield}[endianness=big,bitwidth=1.3em]{32}
    \bitheader{31,25,24,20,19,15,14,12,11,7,6,0} \\
    \bitbox{7}{ funct7                            }
    \bitbox{5}{ rs2                               }
    \bitbox{5}{ rs1                               }
    \bitbox{3}{ funct3                            }
    \bitbox{5}{ rd                                }
    \bitbox{7}{ opcode                            } \\

    \bitbox[]{7}{ 000 0010}
    \bitbox[]{5}{ src2 }
    \bitbox[]{5}{ src1 }
    \bitbox[]{3}{ 011 }
    \bitbox[]{5}{ dest }
    \bitbox[]{7}{ 011 0011 }
  \end{bytefield}
\end{center}
\textbf{Operation:} \texttt{rD = (rs1 <= rs2) ? -1 : 0}


\subsection{p.min rD, rs1, rs2}

Sets rD to the minimum of rs1 and rs2, assuming both are signed 32-bit values.

\begin{center}
  \begin{bytefield}[endianness=big,bitwidth=1.3em]{32}
    \bitheader{31,25,24,20,19,15,14,12,11,7,6,0} \\
    \bitbox{7}{ funct7                            }
    \bitbox{5}{ rs2                               }
    \bitbox{5}{ rs1                               }
    \bitbox{3}{ funct3                            }
    \bitbox{5}{ rd                                }
    \bitbox{7}{ opcode                            } \\

    \bitbox[]{7}{ 000 0010}
    \bitbox[]{5}{ src2 }
    \bitbox[]{5}{ src1 }
    \bitbox[]{3}{ 100 }
    \bitbox[]{5}{ dest }
    \bitbox[]{7}{ 011 0011 }
  \end{bytefield}
\end{center}
\textbf{Operation:} \texttt{rD = rs1 < rs2 ? rs1 : rs2}


\subsection{p.minu rD, rs1, rs2}

Sets rD to the minimum of rs1 and rs2, assuming both are unsigned 32-bit
values.

\begin{center}
  \begin{bytefield}[endianness=big,bitwidth=1.3em]{32}
    \bitheader{31,25,24,20,19,15,14,12,11,7,6,0} \\
    \bitbox{7}{ funct7                            }
    \bitbox{5}{ rs2                               }
    \bitbox{5}{ rs1                               }
    \bitbox{3}{ funct3                            }
    \bitbox{5}{ rd                                }
    \bitbox{7}{ opcode                            } \\

    \bitbox[]{7}{ 000 0010}
    \bitbox[]{5}{ src2 }
    \bitbox[]{5}{ src1 }
    \bitbox[]{3}{ 101 }
    \bitbox[]{5}{ dest }
    \bitbox[]{7}{ 011 0011 }
  \end{bytefield}
\end{center}
\textbf{Operation:} \texttt{rD = rs1 < rs2 ? rs1 : rs2}


\subsection{p.max rD, rs1, rs2}

Sets rD to the maximum of rs1 and rs2, assuming both are signed 32-bit values.

\begin{center}
  \begin{bytefield}[endianness=big,bitwidth=1.3em]{32}
    \bitheader{31,25,24,20,19,15,14,12,11,7,6,0} \\
    \bitbox{7}{ funct7                            }
    \bitbox{5}{ rs2                               }
    \bitbox{5}{ rs1                               }
    \bitbox{3}{ funct3                            }
    \bitbox{5}{ rd                                }
    \bitbox{7}{ opcode                            } \\

    \bitbox[]{7}{ 000 0010}
    \bitbox[]{5}{ src2 }
    \bitbox[]{5}{ src1 }
    \bitbox[]{3}{ 110 }
    \bitbox[]{5}{ dest }
    \bitbox[]{7}{ 011 0011 }
  \end{bytefield}
\end{center}
\textbf{Operation:} \texttt{rD = rs1 > rs2 ? rs1 : rs2}


\subsection{p.maxu rD, rs1, rs2}

Sets rD to the maximum of rs1 and rs2, assuming both are unsigned 32-bit
values.

\begin{center}
  \begin{bytefield}[endianness=big,bitwidth=1.3em]{32}
    \bitheader{31,25,24,20,19,15,14,12,11,7,6,0} \\
    \bitbox{7}{ funct7                            }
    \bitbox{5}{ rs2                               }
    \bitbox{5}{ rs1                               }
    \bitbox{3}{ funct3                            }
    \bitbox{5}{ rd                                }
    \bitbox{7}{ opcode                            } \\

    \bitbox[]{7}{ 000 0010}
    \bitbox[]{5}{ src2 }
    \bitbox[]{5}{ src1 }
    \bitbox[]{3}{ 111 }
    \bitbox[]{5}{ dest }
    \bitbox[]{7}{ 011 0011 }
  \end{bytefield}
\end{center}
\textbf{Operation:} \texttt{rD = rs1 > rs2 ? rs1 : rs2}


\subsection{p.abs rD, rs1}

Computes the absolute value of the signed 32-bit operand rs1.

\begin{center}
  \begin{bytefield}[endianness=big,bitwidth=1.3em]{32}
    \bitheader{31,25,24,20,19,15,14,12,11,7,6,0} \\
    \bitbox{7}{ funct7                            }
    \bitbox{5}{ 00000                             }
    \bitbox{5}{ rs1                               }
    \bitbox{3}{ funct3                            }
    \bitbox{5}{ rd                                }
    \bitbox{7}{ opcode                            } \\

    \bitbox[]{7}{ 000 1010}
    \bitbox[]{5}{ 00000 }
    \bitbox[]{5}{ src1 }
    \bitbox[]{3}{ 000 }
    \bitbox[]{5}{ dest }
    \bitbox[]{7}{ 011 0011 }
  \end{bytefield}
\end{center}
\textbf{Operation:} \texttt{rD = rs1 < 0 ? -rs1 : rs1}


\subsection{p.ror rD, rs1, rs2}

\begin{center}
  \begin{bytefield}[endianness=big,bitwidth=1.3em]{32}
    \bitheader{31,25,24,20,19,15,14,12,11,7,6,0} \\
    \bitbox{7}{ funct7                            }
    \bitbox{5}{ rs2                               }
    \bitbox{5}{ rs1                               }
    \bitbox{3}{ funct3                            }
    \bitbox{5}{ rd                                }
    \bitbox{7}{ opcode                            } \\

    \bitbox[]{7}{ 000 0100}
    \bitbox[]{5}{ src2 }
    \bitbox[]{5}{ src1 }
    \bitbox[]{3}{ 101 }
    \bitbox[]{5}{ dest }
    \bitbox[]{7}{ 011 0011 }
  \end{bytefield}
\end{center}
\textbf{Operation:} \texttt{rD = RotateRight(rs1, rs2)}
