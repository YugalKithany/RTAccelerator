\chapter{Load-Store-Unit (LSU)}
\label{chap:lsu}

The LSU of the core takes care of accessing the data memory. Load and stores on
words (32 bit), half words (16 bit) and bytes (8 bit) are supported.

Table~\ref{tab:lsu_signals} describes the signals that are used by the LSU.


\begin{table}[H]
 \caption{LSU Signals}
 \label{tab:lsu_signals}
  \begin{tabularx}{\textwidth}{@{}llX@{}} \toprule
    \textbf{Signal}               & \textbf{Direction} & \textbf{Description} \\ \toprule
    \signal{data\_req\_o}         & \textbf{output}    & Request ready, must stay high until \signal{data\_gnt\_i} is high for one cycle \\ \hline
    \signal{data\_addr\_o[31:0]}  & \textbf{output}    & Address, sent together with \signal{data\_req\_o} \\ \hline
    \signal{data\_we\_o}          & \textbf{output}    & Write Enable, high for writes, low for reads, sent together with \signal{data\_req\_o} \\ \hline
    \signal{data\_be\_o[3:0]}     & \textbf{output}    & Byte Enable, is set for the bytes to write/read, sent together with \signal{data\_req\_o} \\ \hline
    \signal{data\_wdata\_o[31:0]} & \textbf{output}    & Data to be written to memory, sent together with \signal{data\_req\_o} \\ \hline
    \signal{data\_rdata\_i[31:0]} & \textbf{input}     & Data read from memory, valid when \signal{data\_rvalid\_i} is set \\ \hline
    \signal{data\_rvalid\_i}      & \textbf{input}     & \signal{data\_rdata\_i} is valid. \\ \hline
    \signal{data\_gnt\_i}         & \textbf{input}     & The memory accepted the request, another request can be sent in the next cycle \\ \bottomrule
  \end{tabularx}
\end{table}

\section{Misaligned Accesses}
The LSU is able to perform misaligned accesses, meaning accesses that are not
aligned on natural word boundaries. However it needs to perform two separate
word-aligned accesses internally.
This means that at least two cycles are needed for misaligned loads and stores.


\section{Protocol}
\label{sec:lsu_protocol}

The protocol that is used by the LSU to communicate with a memory works as
follows:

The LSU provides a valid address in \signal{data\_addr\_o} and sets
\signal{data\_req\_o} high. The memory then answers with a \signal{data\_gnt\_i}
set high as soon as it is ready to serve the request. This may happen in the
same cycle as the request was sent or any number of cycles later. After a grant
was received, the address may be changed in the next cycle by the LSU. Also the
\signal{data\_wdata\_o}, \signal{data\_we\_o} and \signal{data\_be\_o} signals
may be changed as it is assumed that the memory has already processed and stored that
information. After the grant, the memory answers with a
\signal{data\_rvalid\_i} set high when \signal{data\_rdata\_i} is valid. This
may happen one cycle after the grant was received, but may take any number of
cycles after the grant was received.
Note that \signal{data\_rvalid\_i} must also be set when a write was performed,
although the \signal{data\_rdata\_i} has no meaning in this case.

Figure~\ref{fig:lsu_trans_basic}, Figure~\ref{fig:lsu_trans_b2b} and
Figure~\ref{fig:lsu_trans_slow} show example timing diagrams of the protocol.

\begin{figure}[H]
  \centering
  \begin{tikztimingtable}
    [timing/d/background/.style={fill=white},
     timing/lslope=0.1,
     xscale=3]

    clk              & 13{C} \\
    data\_addr\_o    & UUU 4D{Address} UU   UU UU \\
    data\_wdata\_o   & UUU 4D{WData} UU     UU UU \\
    data\_req\_o     & LLL HH HH  LL        LL LL \\
    data\_gnt\_i     & LLL LL HH  LL        LL LL \\
    data\_rvalid\_i  & LLL LL LL  HH        LL LL \\
    data\_rdata\_i   & UUU UU UU  2D{RData} UU UU \\
    data\_we\_o      & UUU 4D{WE} UU        UU UU \\
    data\_be\_o      & UUU 4D{BE} UU        UU UU \\
  \end{tikztimingtable}
  \caption{Basic Memory Transaction}
  \label{fig:lsu_trans_basic}
\end{figure}


\begin{figure}[H]
  \centering
  \begin{tikztimingtable}
    [timing/d/background/.style={fill=white},
     timing/lslope=0.1,
     xscale=3]

    clk              & 13{C} \\
    data\_addr\_o    & U 2D{Addr1}  2D{Addr2}  UU UU UU UU \\
    data\_wdata\_o   & U 2D{WData1} 2D{Wdata2} UU UU UU UU \\
    data\_req\_o     & L HH         HH         LL LL LL LL \\
    data\_gnt\_i     & L HH         HH         LL LL LL LL \\
    data\_rvalid\_i  & L LL         HH         HH LL LL LL \\
    data\_rdata\_i   & U UU         2D{RData1} 2D{RData2} UU UU UU\\
    data\_we\_o      & U 2D{WE1}    2D{WE2}    UU UU UU UU \\
    data\_be\_o      & U 2D{BE1}    2D{BE2}    UU UU UU UU \\
  \end{tikztimingtable}
  \caption{Back to Back Memory Transaction}
  \label{fig:lsu_trans_b2b}
\end{figure}

\begin{figure}[H]
  \centering
  \begin{tikztimingtable}
    [timing/d/background/.style={fill=white},
     timing/lslope=0.1,
     xscale=3]

    clk              & 13{C} \\
    data\_addr\_o    & U 6D{Address} UU   UU        UU \\
    data\_wdata\_o   & U 6D{WData} UU     UU        UU \\
    data\_req\_o     & L HH HH HH  LL        LL        LL \\
    data\_gnt\_i     & LLL LL HH  LL        LL        LL \\
    data\_rvalid\_i  & LLL LL LL  LL        HH        UU \\
    data\_rdata\_i   & UUU UU UU  UU        2D{RData} UU \\
    data\_we\_o      & U 6D{WE}   UU        UU        UU \\
    data\_be\_o      & U 6D{BE}   UU        UU        UU \\
  \end{tikztimingtable}
  \caption{Slow Answer Memory Transaction}
  \label{fig:lsu_trans_slow}
\end{figure}


\section{Post-Incrementing Load and Store Instructions}

Post-incrementing load and store instructions perform a load/store operation
from/to the data memory while at the same time increasing the base address by
the specified offset. For the memory access the base address without offset is
used.

Post-incrementing load and stores reduce the number of instructions necessary to
execute when running in a loop, i.e. the address increment can be embedded in
the post-increment instructions. Coupled with the hardware loop extension a
significant reduction in the number of instructions necessary to execute small
loops can be achieved.

A description of the individual instructions that were added can be found below.


\subsection{lb rD, imm(rs1!)}

Loads a byte from data memory on the address \texttt{rs1} and saves \texttt{(rs1
+ I)} in register \texttt{rs1} after the load has completed.

\begin{center}
  \begin{bytefield}[endianness=big,bitwidth=1.3em]{32}
    \bitheader{31,25,24,20,19,15,14,12,11,7,6,0} \\
    \bitbox{12}{ imm[11:0]                        }
    \bitbox{5}{ rs1                               }
    \bitbox{3}{ funct3                            }
    \bitbox{5}{ rd                                }
    \bitbox{7}{ opcode                            } \\

    \bitbox[]{12}{ imm[11:0] }
    \bitbox[]{5}{ src1 }
    \bitbox[]{3}{ 000 }
    \bitbox[]{5}{ dest }
    \bitbox[]{7}{ 000 1011 }
  \end{bytefield}
\end{center}
\textbf{Operation:} \texttt{rD = Sext(mem[rs1]); rs1 += Sext(imm)}


\subsection{lh rD, imm(rs1!)}

Loads a half-word from data memory on the address \texttt{rs1} and saves
\texttt{(rs1 + I)} in register \texttt{rs1} after the load has completed.

\begin{center}
  \begin{bytefield}[endianness=big,bitwidth=1.3em]{32}
    \bitheader{31,25,24,20,19,15,14,12,11,7,6,0} \\
    \bitbox{12}{ imm[11:0]                        }
    \bitbox{5}{ rs1                               }
    \bitbox{3}{ funct3                            }
    \bitbox{5}{ rd                                }
    \bitbox{7}{ opcode                            } \\

    \bitbox[]{12}{ imm[11:0] }
    \bitbox[]{5}{ src1 }
    \bitbox[]{3}{ 001 }
    \bitbox[]{5}{ dest }
    \bitbox[]{7}{ 000 1011 }
  \end{bytefield}
\end{center}
\textbf{Operation:} \texttt{rD = Sext(mem[rs1]); rs1 += Sext(imm)}


\subsection{lw rD, imm(rs1!)}

Loads a word from data memory on the address \texttt{rs1} and saves \texttt{(rs1
+ I)} in register \texttt{rs1} after the load has completed.

\begin{center}
  \begin{bytefield}[endianness=big,bitwidth=1.3em]{32}
    \bitheader{31,25,24,20,19,15,14,12,11,7,6,0} \\
    \bitbox{12}{ imm[11:0]                        }
    \bitbox{5}{ rs1                               }
    \bitbox{3}{ funct3                            }
    \bitbox{5}{ rd                                }
    \bitbox{7}{ opcode                            } \\

    \bitbox[]{12}{ imm[11:0] }
    \bitbox[]{5}{ src1 }
    \bitbox[]{3}{ 010 }
    \bitbox[]{5}{ dest }
    \bitbox[]{7}{ 000 1011 }
  \end{bytefield}
\end{center}
\textbf{Operation:} \texttt{rD = mem[rs1]; rs1 += Sext(imm)}


\subsection{lbu rD, imm(rs1!)}

Loads a byte from data memory on the address \texttt{rs1} and saves \texttt{(rs1
+ I)} in register \texttt{rs1} after the load has completed.

\begin{center}
  \begin{bytefield}[endianness=big,bitwidth=1.3em]{32}
    \bitheader{31,25,24,20,19,15,14,12,11,7,6,0} \\
    \bitbox{12}{ imm[11:0]                        }
    \bitbox{5}{ rs1                               }
    \bitbox{3}{ funct3                            }
    \bitbox{5}{ rd                                }
    \bitbox{7}{ opcode                            } \\

    \bitbox[]{12}{ imm[11:0] }
    \bitbox[]{5}{ src1 }
    \bitbox[]{3}{ 100 }
    \bitbox[]{5}{ dest }
    \bitbox[]{7}{ 000 1011 }
  \end{bytefield}
\end{center}
\textbf{Operation:} \texttt{rD = Zext(mem[rs1]); rs1 += Sext(imm)}


\subsection{lhu rD, imm(rs1!)}

Loads a half-word from data memory on the address \texttt{rs1} and saves
\texttt{(rs1 + I)} in register \texttt{rs1} after the load has completed.

\begin{center}
  \begin{bytefield}[endianness=big,bitwidth=1.3em]{32}
    \bitheader{31,25,24,20,19,15,14,12,11,7,6,0} \\
    \bitbox{12}{ imm[11:0]                        }
    \bitbox{5}{ rs1                               }
    \bitbox{3}{ funct3                            }
    \bitbox{5}{ rd                                }
    \bitbox{7}{ opcode                            } \\

    \bitbox[]{12}{ imm[11:0] }
    \bitbox[]{5}{ src1 }
    \bitbox[]{3}{ 101 }
    \bitbox[]{5}{ dest }
    \bitbox[]{7}{ 000 1011 }
  \end{bytefield}
\end{center}
\textbf{Operation:} \texttt{rD = Zext(mem[rs1]); rs1 += Sext(imm)}



\subsection{lb rD, rs2(rs1!)}

Loads a byte from data memory on the address \texttt{rs1} and saves \texttt{(rs1
+ rs2)} in register \texttt{rs1} after the load has completed.

\begin{center}
  \begin{bytefield}[endianness=big,bitwidth=1.3em]{32}
    \bitheader{31,25,24,20,19,15,14,12,11,7,6,0} \\
    \bitbox{7}{ funct7                            }
    \bitbox{5}{ rs2                               }
    \bitbox{5}{ rs1                               }
    \bitbox{3}{ funct3                            }
    \bitbox{5}{ rd                                }
    \bitbox{7}{ opcode                            } \\

    \bitbox[]{7}{ 000 0000}
    \bitbox[]{5}{ src2 }
    \bitbox[]{5}{ src1 }
    \bitbox[]{3}{ 111 }
    \bitbox[]{5}{ dest }
    \bitbox[]{7}{ 000 1011 }
  \end{bytefield}
\end{center}
\textbf{Operation:} \texttt{rD = Sext(mem[rs1]); rs1 += rs2}


\subsection{lh rD, rs2(rs1!)}

Loads a half-word from data memory on the address \texttt{rs1} and saves
\texttt{(rs1 + rs2)} in register \texttt{rs1} after the load has completed.

\begin{center}
  \begin{bytefield}[endianness=big,bitwidth=1.3em]{32}
    \bitheader{31,25,24,20,19,15,14,12,11,7,6,0} \\
    \bitbox{7}{ funct7                            }
    \bitbox{5}{ rs2                               }
    \bitbox{5}{ rs1                               }
    \bitbox{3}{ funct3                            }
    \bitbox{5}{ rd                                }
    \bitbox{7}{ opcode                            } \\

    \bitbox[]{7}{ 000 1000}
    \bitbox[]{5}{ src2 }
    \bitbox[]{5}{ src1 }
    \bitbox[]{3}{ 111 }
    \bitbox[]{5}{ dest }
    \bitbox[]{7}{ 000 1011 }
  \end{bytefield}
\end{center}
\textbf{Operation:} \texttt{rD = Sext(mem[rs1]); rs1 += rs2}


\subsection{lw rD, rs2(rs1!)}

Loads a word from data memory on the address \texttt{rs1} and saves \texttt{(rs1
+ I)} in register \texttt{rs1} after the load has completed.

\begin{center}
  \begin{bytefield}[endianness=big,bitwidth=1.3em]{32}
    \bitheader{31,25,24,20,19,15,14,12,11,7,6,0} \\
    \bitbox{7}{ funct7                            }
    \bitbox{5}{ rs2                               }
    \bitbox{5}{ rs1                               }
    \bitbox{3}{ funct3                            }
    \bitbox{5}{ rd                                }
    \bitbox{7}{ opcode                            } \\

    \bitbox[]{7}{ 001 0000}
    \bitbox[]{5}{ src2 }
    \bitbox[]{5}{ src1 }
    \bitbox[]{3}{ 111 }
    \bitbox[]{5}{ dest }
    \bitbox[]{7}{ 000 1011 }
  \end{bytefield}
\end{center}
\textbf{Operation:} \texttt{rD = mem[rs1]; rs1 += rs2}


\subsection{lbu rD, rs2(rs1!)}

Loads a byte from data memory on the address \texttt{rs1} and saves \texttt{(rs1
+ rs2)} in register \texttt{rs1} after the load has completed.

\begin{center}
  \begin{bytefield}[endianness=big,bitwidth=1.3em]{32}
    \bitheader{31,25,24,20,19,15,14,12,11,7,6,0} \\
    \bitbox{7}{ funct7                            }
    \bitbox{5}{ rs2                               }
    \bitbox{5}{ rs1                               }
    \bitbox{3}{ funct3                            }
    \bitbox{5}{ rd                                }
    \bitbox{7}{ opcode                            } \\

    \bitbox[]{7}{ 010 0000}
    \bitbox[]{5}{ src2 }
    \bitbox[]{5}{ src1 }
    \bitbox[]{3}{ 111 }
    \bitbox[]{5}{ dest }
    \bitbox[]{7}{ 000 1011 }
  \end{bytefield}
\end{center}
\textbf{Operation:} \texttt{rD = Zext(mem[rs1]); rs1 += rs2}


\subsection{lhu rD, rs2(rs1!)}

Loads a half-word from data memory on the address \texttt{rs1} and saves
\texttt{(rs1 + rs2)} in register \texttt{rs1} after the load has completed.

\begin{center}
  \begin{bytefield}[endianness=big,bitwidth=1.3em]{32}
    \bitheader{31,25,24,20,19,15,14,12,11,7,6,0} \\
    \bitbox{7}{ funct7                            }
    \bitbox{5}{ rs2                               }
    \bitbox{5}{ rs1                               }
    \bitbox{3}{ funct3                            }
    \bitbox{5}{ rd                                }
    \bitbox{7}{ opcode                            } \\

    \bitbox[]{7}{ 010 1000}
    \bitbox[]{5}{ src2 }
    \bitbox[]{5}{ src1 }
    \bitbox[]{3}{ 111 }
    \bitbox[]{5}{ dest }
    \bitbox[]{7}{ 000 1011 }
  \end{bytefield}
\end{center}
\textbf{Operation:} \texttt{rD = Zext(mem[rs1]); rs1 += rs2}



\subsection{sb rs2, imm(rs1!)}

Store a byte to data memory on the address \texttt{rs1} and saves
\texttt{(rs1 + imm)} in register \texttt{rs1} after the store has completed.

\begin{center}
  \begin{bytefield}[endianness=big,bitwidth=1.3em]{32}
    \bitheader{31,25,24,20,19,15,14,12,11,7,6,0} \\
    \bitbox{12}{ imm[11:0]                        }
    \bitbox{5}{ rs1                               }
    \bitbox{3}{ funct3                            }
    \bitbox{5}{ rd                                }
    \bitbox{7}{ opcode                            } \\

    \bitbox[]{12}{ imm[11:0] }
    \bitbox[]{5}{ src1 }
    \bitbox[]{3}{ 000 }
    \bitbox[]{5}{ dest }
    \bitbox[]{7}{ 010 1011 }
  \end{bytefield}
\end{center}
\textbf{Operation:} \texttt{mem[rs1] = rs2[7:0]; rs1 += Sext(imm)}


\subsection{sh rs2, imm(rs1!)}

Store a half-word to data memory on the address \texttt{rs1} and saves
\texttt{(rs1 + imm)} in register \texttt{rs1} after the store has completed.

\begin{center}
  \begin{bytefield}[endianness=big,bitwidth=1.3em]{32}
    \bitheader{31,25,24,20,19,15,14,12,11,7,6,0} \\
    \bitbox{12}{ imm[11:0]                        }
    \bitbox{5}{ rs1                               }
    \bitbox{3}{ funct3                            }
    \bitbox{5}{ rd                                }
    \bitbox{7}{ opcode                            } \\

    \bitbox[]{12}{ imm[11:0] }
    \bitbox[]{5}{ src1 }
    \bitbox[]{3}{ 001 }
    \bitbox[]{5}{ dest }
    \bitbox[]{7}{ 010 1011 }
  \end{bytefield}
\end{center}
\textbf{Operation:} \texttt{mem[rs1] = rs2[15:0]; rs1 += Sext(imm)}


\subsection{sw rs2, imm(rs1!)}

Store a word to data memory on the address \texttt{rs1} and saves
\texttt{(rs1 + imm)} in register \texttt{rs1} after the store has completed.

\begin{center}
  \begin{bytefield}[endianness=big,bitwidth=1.3em]{32}
    \bitheader{31,25,24,20,19,15,14,12,11,7,6,0} \\
    \bitbox{12}{ imm[11:0]                        }
    \bitbox{5}{ rs1                               }
    \bitbox{3}{ funct3                            }
    \bitbox{5}{ rd                                }
    \bitbox{7}{ opcode                            } \\

    \bitbox[]{12}{ imm[11:0] }
    \bitbox[]{5}{ src1 }
    \bitbox[]{3}{ 010 }
    \bitbox[]{5}{ dest }
    \bitbox[]{7}{ 010 1011 }
  \end{bytefield}
\end{center}
\textbf{Operation:} \texttt{mem[rs1] = rs2[31:0]; rs1 += Sext(imm)}



\subsection{sb rs2, rs3(rs1!)}

Store a byte to data memory on the address \texttt{rs1} and saves
\texttt{(rs1 + rs3)} in register \texttt{rs1} after the store has completed.

\begin{center}
  \begin{bytefield}[endianness=big,bitwidth=1.3em]{32}
    \bitheader{31,25,24,20,19,15,14,12,11,7,6,0} \\
    \bitbox{2}{ 00                                }
    \bitbox{5}{ rs3                               }
    \bitbox{5}{ rs2                               }
    \bitbox{5}{ rs1                               }
    \bitbox{3}{ funct3                            }
    \bitbox{5}{ 00000                             }
    \bitbox{7}{ opcode                            } \\

    \bitbox[]{2}{ 00 }
    \bitbox[]{5}{ src3 }
    \bitbox[]{5}{ src2 }
    \bitbox[]{5}{ src1 }
    \bitbox[]{3}{ 100 }
    \bitbox[]{5}{ dest }
    \bitbox[]{7}{ 000 1011 }
  \end{bytefield}
\end{center}
\textbf{Operation:} \texttt{mem[rs1] = rs2[7:0]; rs1 += rs3}


\subsection{sh rs2, rs3(rs1!)}

Store a half-word to data memory on the address \texttt{rs1} and saves
\texttt{(rs1 + rs3)} in register \texttt{rs1} after the store has completed.

\begin{center}
  \begin{bytefield}[endianness=big,bitwidth=1.3em]{32}
    \bitheader{31,25,24,20,19,15,14,12,11,7,6,0} \\
    \bitbox{2}{ 00                                }
    \bitbox{5}{ rs3                               }
    \bitbox{5}{ rs2                               }
    \bitbox{5}{ rs1                               }
    \bitbox{3}{ funct3                            }
    \bitbox{5}{ 00000                             }
    \bitbox{7}{ opcode                            } \\

    \bitbox[]{2}{ 00 }
    \bitbox[]{5}{ src3 }
    \bitbox[]{5}{ src2 }
    \bitbox[]{5}{ src1 }
    \bitbox[]{3}{ 101 }
    \bitbox[]{5}{ dest }
    \bitbox[]{7}{ 000 1011 }
  \end{bytefield}
\end{center}
\textbf{Operation:} \texttt{mem[rs1] = rs2[15:0]; rs1 += rs3}


\subsection{sw rs2, rs3(rs1!)}

Store a word to data memory on the address \texttt{rs1} and saves
\texttt{(rs1 + rs3)} in register \texttt{rs1} after the store has completed.

\begin{center}
  \begin{bytefield}[endianness=big,bitwidth=1.3em]{32}
    \bitheader{31,25,24,20,19,15,14,12,11,7,6,0} \\
    \bitbox{2}{ 00                                }
    \bitbox{5}{ rs3                               }
    \bitbox{5}{ rs2                               }
    \bitbox{5}{ rs1                               }
    \bitbox{3}{ funct3                            }
    \bitbox{5}{ 00000                             }
    \bitbox{7}{ opcode                            } \\

    \bitbox[]{2}{ 00 }
    \bitbox[]{5}{ src3 }
    \bitbox[]{5}{ src2 }
    \bitbox[]{5}{ src1 }
    \bitbox[]{3}{ 110 }
    \bitbox[]{5}{ dest }
    \bitbox[]{7}{ 000 1011 }
  \end{bytefield}
\end{center}
\textbf{Operation:} \texttt{mem[rs1] = rs2[31:0]; rs1 += rs3}
