\chapter{Performance Counters}
\label{chap:perf_count}

Performance Counters in \orion are placed inside the Special-Purpose Registers
and can be accessed with \instr{l.mfspr} and \instr{l.mtspr}.
Figure~\ref{fig:spr_addr} shows the SPR address format and
Table~\ref{tab:pc_spr_addr} shows the respective addresses for configuration and
access to the performance counters.

\begin{table}[H]
 \caption{PC SPR Addresses}
 \label{tab:pc_spr_addr}
 \centering\begin{tabularx}{\textwidth}{@{}ccccX@{}} \toprule
   \textbf{Group \#} & \textbf{Reg \#} & Reg Name     & Access  & Description\\ \toprule
                   7 &  0 - 31         & PCCR0-PCCR31 & R/W     & Performance Counters Count Registers \\ \hline
                   7 &  32             & PCER         & R/W     & Performance Counters Event Register \\ \hline
                   7 &  33             & PCMR         & R/W     & Performance Counters Mode Register \\ \bottomrule
  \end{tabularx}
\end{table}


\section{Performance Counters Mode Register (PCMR)}

\sprDesc{0x3821}{0x0000\_0003}{PCMR}{
  \begin{bytefield}[endianness=big,bitheight=60pt]{32}
    \bitheader{31,1,0} \\
    \bitbox{30}{ Unused                            }
    \bitbox{1}{\rotatebox{90}{\tiny Saturation    }}
    \bitbox{1}{\rotatebox{90}{\tiny Global Enable }}
  \end{bytefield}
}

The \instr{Global Enable} bit controls all performance counters, i.e. if it is set
to \instr{0}, all performance counters are deactivated.
After reset, the \instr{Global Enable} bit is set.

The \instr{Saturation} bit controls saturation behaviour of the performance
counters. If it is set, saturating arithmetic is used.
After reset, the \instr{Saturation} bit is set.

\section{Performance Counters Event Register (PCER)}

\sprDesc{0x3820}{0x0000\_0000}{PCER}{
  \begin{bytefield}[endianness=big,bitheight=60pt]{32}
    \bitheader{31,16,15,14,13,12,11,10,9,8,7,6,5,4,3,2,1,0} \\
    \bitbox{1}{\rotatebox{90}{\tiny (ALL)                 }}
    \bitbox{14}{ Unused                                   }
    \bitbox{1}{\rotatebox{90}{\tiny TCDM CONT             }}
    \bitbox{1}{\rotatebox{90}{\tiny ST\_EXT\_CYC          }}
    \bitbox{1}{\rotatebox{90}{\tiny LD\_EXT\_CYC          }}
    \bitbox{1}{\rotatebox{90}{\tiny ST\_EXT               }}
    \bitbox{1}{\rotatebox{90}{\tiny LD\_EXT               }}
    \bitbox{1}{\rotatebox{90}{\tiny DELAY\_SLOT           }}
    \bitbox{1}{\rotatebox{90}{\tiny BRANCH                }}
    \bitbox{1}{\rotatebox{90}{\tiny JUMP                  }}
    \bitbox{1}{\rotatebox{90}{\tiny ST                    }}
    \bitbox{1}{\rotatebox{90}{\tiny LD                    }}
    \bitbox{1}{\rotatebox{90}{\tiny WBRANCH\_CYC          }}
    \bitbox{1}{\rotatebox{90}{\tiny WBRANCH               }}
    \bitbox{1}{\rotatebox{90}{\tiny IMISS                 }}
    \bitbox{1}{\rotatebox{90}{\tiny JMP\_STALL            }}
    \bitbox{1}{\rotatebox{90}{\tiny LD\_STALL             }}
    \bitbox{1}{\rotatebox{90}{\tiny INSTR                 }}
    \bitbox{1}{\rotatebox{90}{\tiny CYCLES                }}
  \end{bytefield}
}

Each bit in the PCER register controls one performance counter. If the bit is
1, the counter is enabled and starts counting events. If it is 0, the counter
is disabled and its value won't change.

In the ASIC there is only one counter register, thus all counter events are
masked by PCER are ORed together, i.e. if one of the enabled event happens,
the counter will be increased. If multiple non-masked events happen at the same
time, the counter will only be increased by one.

In the FPGA or Simulation version each event has its own counter and can be
accesses separately.


\section{Performance Counters Counter Registers (PCCR0-31)}

\sprDesc{0x3800 - 0x381F}{0x0000\_0000}{PCCR0-31}{
  \begin{bytefield}[endianness=big]{32}
    \bitheader{31,0} \\
    \bitbox{32}{Unsigned integer counter value}
  \end{bytefield}
}

PCCR registers support both saturating and wrap-around arithmetic. This is
controlled by the \instr{saturation} bit in PCMR.

\begin{table}[H]
\begin{tabularx}{\textwidth}{@{}llX@{}} \toprule
  \textbf{Reg Name} & \textbf{Name}       & Description \\ \toprule
  \textbf{PCCR0}  & \textbf{CYCLES}       & Count the number of cycles the core was running \\ \hline
  \textbf{PCCR1}  & \textbf{INSTR}        & Count the number of instructions executed \\ \hline
  \textbf{PCCR2}  & \textbf{LD\_STALL}    & Number of load data hazards \\ \hline
  \textbf{PCCR3}  & \textbf{JMP\_STALL}   & Number of jump register data hazards \\ \hline
  \textbf{PCCR4}  & \textbf{IMISS}        & Cycles waiting for instruction fetches. i.e. the number of instructions wasted due to non-ideal caches \\ \hline
  \textbf{PCCR5}  & \textbf{WBRANCH}      & Number of wrong predicted branches \\ \hline
  \textbf{PCCR6}  & \textbf{WBRANCH\_CYC} & Cycles wasted due to wrong predicted branches \\ \hline
  \textbf{PCCR7}  & \textbf{LD}           & Number of memory loads executed. Misaligned accesses are counted twice \\ \hline
  \textbf{PCCR8}  & \textbf{ST}           & Number of memory stores executed. Misaligned accesses are counted twice \\ \hline
  \textbf{PCCR9}  & \textbf{JUMP}         & Number of jumps (j, jal, jr, jalr)\\ \hline
  \textbf{PCCR10} & \textbf{BRANCH}       & Number of branches (bf, bnf), counts taken and not taken branches\\ \hline
  \textbf{PCCR11} & \textbf{DELAY\_NOP}   & Number of empty (l.nop) delay slots \\ \hline
  \textbf{PCCR12} & \textbf{LD\_EXT}      & Number of memory loads to EXT executed. Misaligned accesses are counted twice. Every non-TCDM access is considered external \\ \hline
  \textbf{PCCR13} & \textbf{ST\_EXT}      & Number of memory stores to EXT executed. Misaligned accesses are counted twice. Every non-TCDM access is considered external \\ \hline
  \textbf{PCCR14} & \textbf{LD\_EXT\_CYC} & Cycles used for memory loads to EXT. Every non-TCDM access is considered external \\ \hline
  \textbf{PCCR15} & \textbf{ST\_EXT\_CYC} & Cycles used for memory stores to EXT. Every non-TCDM access is considered external \\ \hline
  \textbf{PCCR16} & \textbf{TCDM\_CONT}   & Cycles wasted due to TCDM/log-interconnect contention \\ \hline
  \textbf{PCCR31} & \textbf{ALL}          & Special Register, a write to this register will set all counters to the supplied value\\ \bottomrule
\end{tabularx}
\end{table}

In the FPGA, RTL simulation and Virtual-Platform there are individual counters
for each event type, i.e. PCCR0-30 each represent a separate register.
To save area in the ASIC, there is only one counter and one counter register.
Accessing PCCR0-30 will access the same counter register in the ASIC.
Reading/writing from/to PCCR31 in the ASIC will access the same register as
PCCR0-30.

Figure~\ref{fig:events} shows how events are first masked with the PCER register
and then ORed together to increase the one performance counter PCCR.

\begin{figure}[H]
  \centering
  \includegraphics[width=0.5\textwidth]{./figures/events}
  \caption{Events and PCCR, PCMR and PCER on the ASIC.}
  \label{fig:events}
\end{figure}
